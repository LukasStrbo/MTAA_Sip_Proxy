%%%%%%%%%%%%%%%%%%%%%%%%%%%%%%%%%%%%%%%%%
% University/School Laboratory Report
% LaTeX Template
% Version 3.1 (25/3/14)
%
% This template has been downloaded from:
% http://www.LaTeXTemplates.com
%
% Original author:
% Linux and Unix Users Group at Virginia Tech Wiki 
% (https://vtluug.org/wiki/Example_LaTeX_chem_lab_report)
%
% License:
% CC BY-NC-SA 3.0 (http://creativecommons.org/licenses/by-nc-sa/3.0/)
%
%%%%%%%%%%%%%%%%%%%%%%%%%%%%%%%%%%%%%%%%%

%----------------------------------------------------------------------------------------
%	PACKAGES AND DOCUMENT CONFIGURATIONS
%----------------------------------------------------------------------------------------

\documentclass[10pt,oneside,slovak,a4paper]{article}

\usepackage{graphicx} % Required for the inclusion of images
\usepackage{amsmath} % Required for some math elements 
\usepackage[utf8]{inputenc}
\usepackage[slovak]{babel}
\usepackage{listings}
\usepackage{indentfirst}
\usepackage{float}
\usepackage{fancyhdr}
\usepackage{enumitem}
\usepackage[a4paper, total={6in, 8in}]{geometry}
\usepackage[table,xcdraw]{xcolor}
\usepackage{verbatim}
\usepackage{titling}
\usepackage{hyperref}
%\usepackage{rotating} %PRIDANY PACKAGE
%\usepackage{booktabs} %PRIDANY PACKAGE
%\usepackage{multirow}%PRIDANY PACKAGE
%\usepackage{afterpage}
%\usepackage{lipsum}
%\usepackage{minted}

\pagestyle{fancy}


\graphicspath{ {./images/} }
%\pagestyle{myheadings}

\setlength{\droptitle}{-5ex} %posunutie Nazvu autora a pod vyssie

\fancypagestyle{plain}{%
   \fancyhf{} 
   \fancyfoot[C]{\tiny  Autor predlohy : Linux and Unix Users Group at Virginia Tech Wiki. Názov predlohy : University/School Laboratory Report. Predloha dokumentu licencovaná pod CC BY-NC-SA 3.0 (http://creativecommons.org/licenses/by-nc-sa/3.0/).}
   \renewcommand{\headrulewidth}{0pt}
}

%\usepackage{times} % Uncomment to use the Times New Roman font

%----------------------------------------------------------------------------------------
%	DOCUMENT INFORMATION
%----------------------------------------------------------------------------------------

\title{\textsc{Mobilné technológie a aplikácie}\\ Dokumentácia k zadaniu} % Title

\author{Lukáš \textsc{Štrbo}} % Author name

\date{1. marec 2022}
%\date{\today} % Date for the report

\author{Lukáš \textsc{Štrbo}\\[2pt]
	{\small Slovenská technická univerzita v Bratislave}\\
	{\small Fakulta informatiky a informačných technológií}\\
	{\small \texttt{xstrbol@stuba.sk}}\\
{\small \texttt{ID: 110903}}\\
	}


\begin{document}

\maketitle % Insert the title, author and date

\begin{center}
\begin{tabular}{l r}
Zadanie č. 1 : & SIP Proxy (telefónna ústredňa)\\ 
Cvičiaci: & Ing. Marek Galinski, PhD.
\end{tabular}
\end{center}

\section{Zadanie}
Sprevádzkovanie SIP Proxy, ktorá umožní prepájanie a realizáciu hovorov medzi štandardnými SIP klientami.

Na implementáciu SIP Proxy si môžete zvoliť akýkoľvek programovací jazyk a použiť akúkoľvek SIP knižnicu, ktorá pre daný programovací jazyk existuje. Vo výsledku však musíte spúšťať “váš kód”, v ktorom sú zakomponované knižnice, ktoré poskytujú funkcionalitu SIP Proxy. Hovor musí byť realizovaný medzi dvomi fyzickými zariadeniami v rámci LAN siete


\textbf{Povinné funkcionality}
\begin{enumerate}[label=(\alph*)]
	\item Registrácia účastníka (bez nutnosti autentifikácie) 
	\item Vytočenie hovoru a zvonenie na druhej strane 
	\item Prijatie hovoru druhou stranou, fungujúci hlasový hovor 
	\item Ukončenie hlasového hovoru (prijatého aj neprijatého) 
\end{enumerate}


\textbf{Doplnkové funkcionality}
\begin{enumerate}[label=(\alph*)]
	\item Možnosť zrealizovať konferenčný hovor (aspoň 3 účastníci) 
	\item Možnosť presmerovať hovor 
	\item Možnosť realizovať videohovor 
	\item Logovanie “denníka hovorov”
	\item Úprava SIP stavových kódov v zdrojovom kóde proxy
\end{enumerate}

%\thispagestyle{fancy}
\section{Záver}


\bibliographystyle{plain}
\bibliography{zdroje}
\end{document}

%\begin{enumerate}[label=\textbf{\arabic*})]
%	\item X
%\end{enumerate}

%\begin{enumerate}[label=(\alph*)]
%	\item X
%\end{enumerate}

%\begin{figure}[H]
%	\centering
%	\includegraphics[scale=1, width=\textwidth]{FrameTypes.drawio.pdf}
%	\caption{Typy rámcov}
%\end{figure}